\documentclass{article}
\usepackage{geometry}
\geometry{a4paper, margin=1in}
\usepackage{enumitem}
\usepackage{hyperref}
\hypersetup{
    colorlinks=true,
    linkcolor=blue,
    filecolor=magenta,      
    urlcolor=cyan,
}
\usepackage{graphicx}
\graphicspath{ {./images/} }

\title{Recess Weekly Report 3}
\author{Nathaniel Mugenyi \\ Student Registration No: M24B23/027}
\date{Submission Date: 2, Feb, 2025 \\ Instructors: Mr. Solomon Opio, Ms. Tryphine, Mr. Kasole Ahmed}

\begin{document}

\begin{titlepage}
    \centering
    \includegraphics[scale=0.7]{images/UCU .png} 
    \vspace{1cm}
    
    {\LARGE \textbf{Recess Weekly Report 3} \par}
    \vspace{1cm}
    
    {\large \textbf{Nathaniel Mugenyi} \par}
    {\large Student Registration No: M24B23/027 \par}
    \vspace{1cm}
    
    {\large Submission Date: 2, Feb, 2025 \par}
    {\large Instructors: Mr. Solomon Opio, Ms. Tryphine, Mr. Kasole Ahmed \par}
    \vfill
\end{titlepage}

\section*{1. Title Page}


\section*{2. Executive Summary}
This week was a blend of ideation, skill development, and practical application, all aimed at advancing our project and personal growth. We began by refining our project idea, \textit{Career Campus}, by addressing key pain points and improving our pitch deck. A significant milestone was the creation of a Figma prototype, which allowed us to visualize the user experience and gather feedback. We also had the opportunity to pitch our idea at Motive, where we received valuable insights into aligning our solution with market needs.

In addition to technical work, we engaged in sessions on emotional intelligence and preventive maintenance, which provided a well-rounded learning experience. The temperament test revealed my personality type as Melancholic, offering insights into my strengths and areas for growth. The week concluded with an assignment to build a website for a computer repair store, allowing us to apply our technical skills in a real-world context. Overall, the week was highly productive, with significant progress made in both project development and personal skill enhancement.

\section*{3. Objectives of the Week}
\begin{enumerate}
    \item \textbf{Objective 1:} Refine the project idea by addressing pain points and improving the pitch deck to ensure clarity and impact.
    \item \textbf{Objective 2:} Develop a Figma prototype for \textit{Career Campus} to create a tangible representation of the user interface and experience.
    \item \textbf{Objective 3:} Gain insights into product-market fit, emotional intelligence, and preventive maintenance to enhance both technical and interpersonal skills.
\end{enumerate}

\section*{4. Activities Undertaken}

\subsection*{Activity 1: Refining the Project Idea and Pitch Deck}
\begin{itemize}
    \item \textbf{Description:} We conducted a thorough review of our project idea, \textit{Career Campus}, to identify and address key pain points. This involved brainstorming sessions, user feedback analysis, and iterative improvements to the pitch deck.
    \item \textbf{Tools/Resources Used:}
    \begin{itemize}
        \item Canva for creating and updating the pitch deck.
        \item Team discussions and feedback from peers and instructors.
    \end{itemize}
    \item \textbf{Challenges Encountered:}
    \begin{itemize}
        \item Balancing brevity with comprehensiveness in the pitch deck.
        \item Ensuring the pain points addressed were relevant and impactful.
    \end{itemize}
    \item \textbf{Learning Insights:}
    \begin{itemize}
        \item A well-structured pitch deck is essential for effectively communicating the value proposition.
        \item Iterative refinement is key to addressing user needs and improving the project.
    \end{itemize}
\end{itemize}

\subsection*{Activity 2: Building a Figma Prototype for \textit{Career Campus}}
\begin{itemize}
    \item \textbf{Description:} We designed a Figma prototype to visualize the user interface and experience of \textit{Career Campus}. This included creating wireframes, defining user flows, and incorporating feedback from team members.
    \item \textbf{Tools/Resources Used:}
    \begin{itemize}
        \item Figma for prototyping and collaboration.
        \item User personas and scenarios to guide the design process.
    \end{itemize}
    \item \textbf{Challenges Encountered:}
    \begin{itemize}
        \item Ensuring the prototype was both functional and visually appealing.
        \item Incorporating feedback from multiple stakeholders.
    \end{itemize}
    \item \textbf{Learning Insights:}
    \begin{itemize}
        \item Prototyping is an iterative process that requires constant refinement.
        \item User-centered design is critical for creating effective solutions.
    \end{itemize}
\end{itemize}

\subsection*{Activity 3: Pitching at Motive and Learning About Product-Market Fit}
\begin{itemize}
    \item \textbf{Description:} We visited Motive to pitch our project idea and received constructive feedback. Additionally, we attended a session on product-market fit, which emphasized the importance of aligning our solution with market needs.
    \item \textbf{Tools/Resources Used:}
    \begin{itemize}
        \item Pitch deck for the presentation.
        \item Screen \& Laptop for presenting.
    \end{itemize}
    \item \textbf{Challenges Encountered:}
    \begin{itemize}
        \item Adapting the pitch to resonate with different audiences.
        \item Understanding and applying the concept of product-market fit.
    \end{itemize}
    \item \textbf{Learning Insights:}
    \begin{itemize}
        \item A deep understanding of the target market is essential for product success.
        \item Feedback from diverse perspectives can significantly improve the project.
    \end{itemize}
\end{itemize}

\subsection*{Activity 4: Emotional Intelligence and Temperament Test}
\begin{itemize}
    \item \textbf{Description:} Ms. Typhyn conducted a session on emotional intelligence, highlighting its importance in teamwork and leadership. We also took a temperament test, which revealed my personality type as Melancholic.
    \item \textbf{Tools/Resources Used:}
    \begin{itemize}
        \item Temperament test questionnaire.
        \item SWOT analysis framework for self-evaluation.
    \end{itemize}
    \item \textbf{Challenges Encountered:}
    \begin{itemize}
        \item Interpreting the results of the temperament test and applying them to personal growth.
        \item Conducting an honest and thorough SWOT analysis.
    \end{itemize}
    \item \textbf{Learning Insights:}
    \begin{itemize}
        \item Emotional intelligence is a critical skill for effective collaboration and leadership.
        \item Self-awareness is the first step toward personal and professional development.
    \end{itemize}
\end{itemize}

\subsection*{Activity 5: Preventive Maintenance and Troubleshooting}
\begin{itemize}
    \item \textbf{Description:} Mr. Ahmed led a session on preventive maintenance and troubleshooting, providing practical tips for maintaining systems and resolving issues. We were then assigned to build a website for a computer repair store.
    \item \textbf{Tools/Resources Used:}
    \begin{itemize}
        \item Troubleshooting guides and manuals.
        \item Web development tools (e.g., HTML, CSS, TypeScript, Next.Js).
    \end{itemize}
    \item \textbf{Challenges Encountered:}
    \begin{itemize}
        \item Applying theoretical knowledge to practical tasks.
        \item Ensuring the website was both functional and user-friendly.
        \item Learning a new programming language.
        \item Working with the team remotely on the weekend.
    \end{itemize}
    \item \textbf{Learning Insights:}
    \begin{itemize}
        \item Preventive maintenance can significantly reduce system failures and downtime.
        \item Practical assignments are an effective way to reinforce technical skills.
    \end{itemize}
\end{itemize}

\section*{5. Learning Outcomes}
\begin{itemize}
    \item \textbf{Technical Skills Developed:}
    \begin{itemize}
        \item Figma prototyping and user interface design.
        \item Web development basics, including HTML and CSS.
        \item Troubleshooting and preventive maintenance techniques.
    \end{itemize}
    \item \textbf{Conceptual Understanding:}
    \begin{itemize}
        \item The importance of product-market fit in product development.
        \item Emotional intelligence and its role in teamwork and leadership.
    \end{itemize}
    \item \textbf{Methodological Insights:}
    \begin{itemize}
        \item Iterative design and prototyping for user-centered solutions.
        \item SWOT analysis for self-evaluation and project improvement.
    \end{itemize}
\end{itemize}

\section*{6. Reflections}
This week was both challenging and rewarding, offering a mix of technical and interpersonal learning opportunities. The most interesting aspect was the session on emotional intelligence, which provided valuable insights into self-awareness and team dynamics. The most challenging part was refining the pitch deck to ensure it resonated with different audiences, as it required balancing brevity with comprehensiveness.

These activities significantly contributed to the overall workshop goals by enhancing both technical and interpersonal skills. For future sessions, incorporating more hands-on exercises and real-world case studies could further enrich the learning experience. Additionally, more time for peer feedback and collaboration would help solidify the concepts learned.

\section*{7. Action Plan for Next Week}
\begin{itemize}
    \item \textbf{Review Materials:} Revisit notes on product-market fit, emotional intelligence, and preventive maintenance.
    \item \textbf{Skill Development:} Practice advanced Figma features and web development techniques, such as responsive design and JavaScript.
    \item \textbf{Preparatory Tasks:} Complete the website assignment for the computer repair store and prepare for the next pitch presentation.
\end{itemize}

\section*{8. Appendices (Optional)}
\begin{itemize}
    \item Canva Pitch Deck: \url{https://shorturl.at/nc9K0}
    \item Link for Website: \url{https://codexrepairs.vercel.app/}
\end{itemize}

\section*{9. References}
\begin{itemize}
    \item \url{https://www.typescriptlang.org/}
    \item \url{https://www.nextjs.com/}
\end{itemize}

\end{document}
